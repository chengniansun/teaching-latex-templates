\documentclass[sigplan,screen]{acmart}

\settopmatter{printacmref=false}
\settopmatter{printacmref=false} % Removes citation information below abstract
\renewcommand\footnotetextcopyrightpermission[1]{} % removes footnote with conference information in first column
\pagestyle{plain} % removes running headers


\usepackage{enumitem}

\begin{document}

% List the week, the paper id, and the title
\title{Proposal for CS499T Spring 2022}
\subtitle{Your Title Here}


\author{Student: Your Name }
\affiliation{
  \institution{Cheriton School of Computer Science\\University of Waterloo}
  \state{Ontario}
  \country{Canada}
}
\email{student@uwaterloo.ca}


\author{Instructor: Chengnian Sun}
\affiliation{
    \institution{Cheriton School of Computer Science\\University of Waterloo}
    \state{Ontario}
    \country{Canada}
}
\email{cnsun@uwaterloo.ca}


\begin{abstract}
    A short paragraph here to summarize what you want to do here.
\end{abstract}


\maketitle



\section{Introduction}
You can talk about the motivation, the problem briefly,
and what you want to do briefly.


\section{Outcome}
Describe what you plan to do in this course.

\section{Milestones}
This section outlines the milestones for (Winter/Spring/Fall) 2022.

\begin{description}[leftmargin=0mm]
\item[Week 1] what to do
\item[Week 2] what to do
\item[Week ...] what to do
\end{description}

e.g., importance of the problem, novelty, evaluation

\section{Related Work}
You need to summarize the papers you read in CS499R.

For example, in~\cite{tkfuzz}, the authors.....


\section{Conclusion}
Just a short paragraph to conclude the proposal.

\bibliographystyle{ACM-Reference-Format}
\bibliography{main}

\end{document}
\endinput
